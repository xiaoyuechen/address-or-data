\documentclass[a4paper]{article}
\usepackage{cite}
\usepackage{listings}
\lstset{basicstyle=\ttfamily}
\usepackage{booktabs}
\usepackage{hyperref}
\usepackage{pdflscape}

\title{Distinguishing Between Addresses and Data\\Thesis Specification}

\author{Xiaoyue Chen}

\begin{document}
\maketitle

\section{Background}
% Here you describe in what context your thesis is to be done. What
% prerequisites are valid, what is the goal of the project from the
% supervisors point of view, what is available and has been done
% before, under what circumstances should the work be done.
Most modern computers do not distinguish between memory addresses and
data. Any content in the memory could be used directly or indirectly
as an address to access the memory. Even if the content is just data
(e.g., a password hash), it could still be used to reference the
memory (e.g., as an array index). While this property makes the
architecture more flexible, it also raises security concerns. By
exploiting this property, attackers could break memory isolation
between processes and leak confidential information.

Distinguishing between memory addresses and data opens opportunities
for more secure systems. The processor could forbid loads that use
data as address to prevent such attacks. On the other hand,
separating address from data can lead to more efficient address
translation.

\subsection{Spectre and Meltdown}
Spectre \cite{kocher2019spectre} and Meltdown \cite{lipp2018meltdown}
relies on using secret data to reference the memory. In both attacks,
attackers first access the secret data which will cause invalid
address accessing exception. Due to modern processors' out-of-order
execution and speculative execution optimisations, transient
instructions whose architectural effects will not be committed are
also executed.

In the transient instructions, attackers access the memory by using
the secret data as memory addresses. Although the transient
instructions will have no architectural effects, the memory accesses
would leave microarchitectural side effects, i.e., the referenced
memory is stored in the cache. Finally, attackers could use a cache
side channel to observe the side effects and read the secret data by
exploiting timing differences.

\subsection{Speculative taint tracking (STT)}
Speculative taint tracking (STT) introduces a framework to protect
speculatively accessed data from speculative execution
attacks~\cite{yu2019speculative}. By blocking side channels and using
taint/untaint procedure to wake instructions up early, it disables
protection on previously protected data, as soon as doing so is safe.
STT is the current state of art novel protection framework.

\subsection{Effects of memory addresses/data distinction}
The attacks introduced above could be mitigated if memory addresses
and data are distinguished. Using data as memory addresses could be
forbidden. As a result, the transient instructions in the attacks
would leave no microarchitectural side effects. This will improve the
security of computer systems.

\subsection{Goal}
The goal is to develop a tool to distinguish between memory address
and data based on the ideas from STT. The tool should dynamically mark
memory regions of programs as memory address or data. The tool should
also be able to collect data that characterize memory access of
programs, e.g., number of untaits.

\section{Description of the task}
Develop a tool to distinguish memory address from data. Characterize
memory access patterns of programs.

\section{Methods}
Use Intel pin tool to dynamically inject and replace instructions of
programs to perform dynamic information flow tracking (DIFT).

Tracking the number of untaints of programs over their lifetime based
on STT to characterize memory access of programs.

\section{Relevant courses}
\begin{itemize}
\item Advanced Computer Architecture
\item Computer Architecture I
\item Compiler Design I
\item Operating Systems I
\item Secure Computer Systems I
\end{itemize}

\section{Delimitations}
The thesis project aims to offer a characterization of memory access
of programs relating to speculative execution. The project does not
aim to propose or implement mitigation methods to protect against
speculative execution attacks.

\section{Time plan}
Xiaoyue Chen will be working full-time on the project for the entire
spring term in 2022. The project starts on 12 Jan (week 2), and ends
on 12 Jun (week 24). The project should be carried out in an iterative
approach, i.e., development, experiment, and writing. A time plan
follows table~\ref{tab:timeplan}.

\begin{landscape}
\begin{table}[h]
  \centering
  \begin{tabular}{lll}
    \toprule
    Week &  Tasks & Write report sections \\
    \midrule
    2,3 & Bibliographical research, searching and learning software tools & Introduction and background \\
    4,5,6 & Develop software address-tainting system using pin tool & Background \\
    7 & Running experiments and analyzing results & Methods \\
    8,9,10 & Further developments and testing & Methods \\
    11 & Experiments and mid-project evaluation & Results \\
    12,13 & Experiments, logging untaint and plotting & Results \\
    14,15,16 & Experiments and iterative report writing & Results and discussion \\
    17,18 & Free slot & Results and discussion \\
    19.20 & Receiving feedback and report writing & All \\
    21 & Finalizing report & All \\
    22,23 & Book and prepare for presentation & --- \\
    24 & Project ends & --- \\
    \bottomrule
  \end{tabular}
  \caption{The time plan for the project}
  \label{tab:timeplan}
\end{table}
\end{landscape}


\bibliographystyle{IEEEtran} \bibliography{references}

\end{document}

%%% Local Variables:
%%% mode: latex
%%% TeX-master: t
%%% End:
